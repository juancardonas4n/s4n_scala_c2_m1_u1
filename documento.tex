\documentclass[12pt,letterpaper]{article}

\usepackage[spanish]{babel}
\usepackage[utf8]{inputenc}
\usepackage{pgf}
\usepackage{tikz}
%\usepackage{algorithmicx}
\usepackage{algpseudocode}
\usepackage{color}
\usepackage{amsmath}
% \usepackage{bbm}
\usepackage{txfonts}
\usepackage{url}

% \usepackage{pgflibraryarrows}
% \usepackage{pgflibrarysnakes}
% \usepackage{pgflibraryshapes}
\usepackage{alltt}
\usepackage{enumerate}
\usetikzlibrary{automata,snakes,backgrounds,trees}

\begin{document}

\begin{eqnarray*}
  n! = n \times (n - 1) \times (n - 2) \times \ldots \times 3 \times 2 \times 1
\end{eqnarray*}

\[
  factorial(n) =
  \begin{cases}
    n = 0 , & 1 \\
    \text{En otro caso}, & n * factorial(n - 1) \\
  \end{cases}
\]

\begin{eqnarray*}
  n = 0 & = & 1\\
  n = 1 & = & 1 \\
  n = 2 & = & 2 \times 1\\
  n = 3 & = & 3 \times 2 \times 1 \\
  n = 4 & = & 4 \times 3 \times 2 \times 1 \\
  \ldots & \ldots & \ldots \\
  n     & = & n \times (n - 1) \times (n - 2) \times \ldots \times 3 \times 2 \times 1 \\
\end{eqnarray*}



\begin{eqnarray*}
  n = 0 & = & 1\\
  n = 1 & = & 1 \times factorial(0) \\
  n = 2 & = & 2 \times factorial(1)\\
  n = 3 & = & 3 \times factorial(2) \\
  n = 4 & = & 4 \times factorial(3) \\
  \ldots & \ldots & \ldots \\
  n     & = & n \times factorial(n - 1) \\
\end{eqnarray*}


% \begin{eqnarray*}
%   a^0 & = & 1 = 1 \\
%   a^1 & = & 1 \times a \\
%   a^2 & = & 1 \times a \times a\\
%   a^3 & = & 1 \times a \times a \times a \\
%   \vdots & & \vdots \\
%   a^n & = & 1 \times a \times a \times a \times \ldots \times a \\
% \end{eqnarray*}

% \begin{eqnarray*}
%   a^0 & = & 1  \\
%   a^1 & = & 1 \times a^0   \\
%   a^2 & = & 1 \times a \times a^1 \equiv a \times a^1  \\
%   a^3 & = & a \times a^2 \\
%   a^4 & = & a \times a^3 \\
%   \vdots & & \vdots \\
%   a^n & = & a \times a^{n-1} \\
% \end{eqnarray*}


% \begin{eqnarray*}
%   r^2 \leq n \wedge n < (r + a)^2
% \end{eqnarray*}


% \[
%   iRaizEntera(n,a) =
%   \begin{cases}
%     a^2 > n, & 0 \\
%     a^2 \leq n, & r' = iRaizEntera(n,2a)
%     \begin{cases}
%       n < (r' + a)^2, & r' \\
%       n \leq (r' + a)^2, & r' + a \\
%     \end{cases}
%   \end{cases}
% \]

% \[
%   ln(x) = \frac{(x-1)}{1} - \frac{(x-1)^2}{2} + \frac{(x-1)^3}{3} - \frac{(x-1)^4}{4} + \ldots 
% \]

% \[
%   e^x = \sum^{\infty}_{n=1}{\frac{x^n}{n!}} \equiv 1 + x + \frac{x^2}{2!} + \frac{x^3}{3!} + \ldots
% \]

% \[
%   \sqrt{x} = \sqrt{a} + \frac{(x - a)}{1}\frac{1}{2\sqrt{a}} - 
% \]


% \begin{algorithmic}
%   %\qocaption{Algoritmo babilónico}\label{algobabel}
%   %\begin{algorithmic}[1]
%     \Function{raiz}{$x$}
%     \State {$r \gets x$}
%     \State {$t \gets 0$}
%     \While{$t \neq r$}
%     \State {$t \gets r$}
%     \State {$r \gets \frac{1}{2}(\frac{x}{r} + r)$}
%     \EndWhile
%     \State \textbf{return} $r$
%     \EndFunction
%   %\end{algorithmic}
% \end{algorithmic}


% \[
%   a^n =  \begin{cases}
%     a, & \text{Si }n=1\\
%     (a^{\frac{n}{2}})^2 & \text{Si }n\ \text{es par} \\
%     a \times a^{n-1} & \text{En caso contrario} \\
%  \end{cases}
% \]

% \[
%   A(m,n) = \begin{cases}
%     n + 1, & \text{Si}\ m=0\\
%     A(m-1,1) & \text{Si}\ m>0\ \text{y}\ n=0\\
%     A(m-1,A(m,n-1) & \text{Si}\ m>0\ \text{y}\ n>0\\
%     \end{cases}
% \]
\end{document}

%%% Local Variables:
%%% mode: latex
%%% TeX-master: t
%%% End:
